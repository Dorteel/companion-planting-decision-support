%% The first command in your LaTeX source must be the \documentclass command.
%%
%% Options:
%% twocolumn : Two column layout.
%% hf: enable header and footer.
\documentclass[
% twocolumn,
% hf,
]{ceurart}

%%
%% One can fix some overfulls
\sloppy

%%
%% Minted listings support 
%% Need pygment <http://pygments.org/> <http://pypi.python.org/pypi/Pygments>
\usepackage{listings}
%% auto break lines
\lstset{breaklines=true}

\usepackage{cleveref}
\usepackage{todonotes}

\newcommand{\patrick}[1]{\todo[inline]{Patrick: #1}}

%%
%% end of the preamble, start of the body of the document source.
\begin{document}

%%
%% Rights management information.
%% CC-BY is default license.
\copyrightyear{2024}
\copyrightclause{Copyright for this paper by its authors.
  Use permitted under Creative Commons License Attribution 4.0
  International (CC BY 4.0).}

%%
%% This command is for the conference information
\conference{The second international Workshop on Knowledge Graphs for Sustainability --- KG4S}

%%
%% The "title" command
\title{Companion Plant Paper}



\author{Patrick Koopmann}[%
orcid=0000-0001-5999-2583,
email=p.k.koopmann@vu.nl,
url=https://pkoopmann.github.io/,
]
% \fnmark[1]
% \cormark[1]

\author{Ilaria Tiddi}[%
orcid=0000-0001-7116-9338,
email=i.tiddi@vu.nl,
url=https://kmitd.github.io/ilaria/,
]
% \fnmark[1]
% \cormark[1]
\address{Vrije Universiteit Amsterdam, De Boelelaan 1105, 1081 HV Amsterdam, The Netherlands}

%% Footnotes
% \cortext[1]{Corresponding author.}
% \fntext[1]{These authors contributed equally.}

\patrick{Other authors: add your own todo macros and author information}
\patrick{Come up with a title}

%%
%% The abstract is a short summary of the work to be presented in the
%% article.
\begin{abstract}
TODO
\end{abstract}

%%
%% Keywords. The author(s) should pick words that accurately describe
%% the work being presented. Separate the keywords with commas.
\begin{keywords}
  TODO
\end{keywords}

%%
%% This command processes the author and affiliation and title
%% information and builds the first part of the formatted document.
\maketitle

\section{Introduction}

\section{Related Work}

\section{The Ontology}

\begin{itemize}
 \item overview
 \item different subsections for different components: core ontology,
 generated information (and generation process), extracted complex axioms
\end{itemize}


\section{Use Cases}

\subsection{Finding (Anti-)Companions}

If we have a specific plant, and want to look for (anti-)companinons, these can be
directly queried using a subclass query, already in Prot\'eg\'e, using the class expression
\verb|companion some PLANT| respectively \verb|anti_companion some PLANT|, where PLANT is
the identified of the corresponding plant. For each plant, we can furthermore ask for detailed
explanations using the functionality of Prot\'eg\'e (see \Cref{fig:justification-companion}).
All relevant axioms are in the OWL EL profile, more
specifically in the subset that is supported by the ELK reasoner\cite{ELK}. As a consequence,
we can use more advanced explanation services based on proofs that are obtained by the \ELK
reasoner, such as the ones provided by XXX and evee~\cite{evee} (\Cref{fig:proof-companion}).

\subsection{Analysing Garden Configurations}

The second use case regards the actual garden---assuming the user has decided which

\section{Prototypical User Interface}

\section{Conclusion}



\begin{acknowledgments}
  TODO
\end{acknowledgments}

%%
%% Define the bibliography file to be used
\bibliography{kgs-bibliography}

%%
%% If your work has an appendix, this is the place to put it.
\appendix


\end{document}

%%
%% End of file
